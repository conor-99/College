\documentclass[12pt]{article}
\usepackage{amsmath}
\usepackage{geometry}
\usepackage{graphicx}
\usepackage{hyperref}
\usepackage[latin1]{inputenc}
\usepackage{listings}
\usepackage[dvipsnames]{xcolor}
\renewcommand{\labelitemi}{$\textendash$}
\geometry{
    a4paper,
    total={170mm,257mm},
    left=20mm,
    right=20mm,
    top=5mm,
    bottom=15mm
}

\title{CS4004: Assignment 1}
\author{Conor McCauley - 17323203}
\date{November 3, 2020}

\begin{document}

\maketitle

\section*{Question 1}

\noindent (a) As no contradictions occurred we know that the following semantic entailment is falsifiable and therefore it does not hold:

\begin{center}
    \begin{tabular}{cccccccccc}
        & $\neg p$ & $\vee$ & $(q$ & $\to$ & $p)$ & $\models$ & $\neg p$ & $\wedge$ & $q$ \\
        1   &   & T &   &   &   &   &   & F &   \\
        1.1 & T & T & F & T & F &   & T & F & F \\
        1.2 & F & T & T & T & T &   & F & F & T \\
        1.3 & F & T & F & T & T &   & F & F & F \\
    \end{tabular}
\end{center}

\noindent (b) As a contradiction occurred we know that the following semantic entailment is not falsifiable and therefore it does hold:

\begin{center}
    \begin{tabular}{ccccccccc}
        & $\models$ & $((A$ & $\to$ & $B)$ & $\to$ & $A)$ & $\to$ & $A$ \\
        1   &   &   &   &   &   &   & F &    \\
        1.1 &   & \textcolor{red}{T} & F & F & T & F & F & \textcolor{red}{F}  \\
    \end{tabular}
\end{center}

\section*{Question 2}

\noindent (a)

\framebox{\parbox{\dimexpr\linewidth-0\fboxsep-0\fboxrule}{\itshape%
    $1.$ $\neg (A_1 \vee A_2),$ $premise$ \\
    \framebox{\parbox{\dimexpr\linewidth-2\fboxsep-2\fboxrule}{\itshape%
        $2.$ $A_1,$ $assumption$ \\
        $3.$ $A_1 \vee A_2,$ $\vee i_1 (2)$ \\
        $4.$ $\bot,$ $\neg e (1, 3)$
    }}
    \framebox{\parbox{\dimexpr\linewidth-2\fboxsep-2\fboxrule}{\itshape%
        $5.$ $A_2,$ $assumption$ \\
        $6.$ $A_1 \vee A_2,$ $\vee i_2 (5)$ \\
        $7.$ $\bot,$ $\neg e (1, 6)$
    }}
    $8.$ $\neg A_1,$ $\neg i (2-4)$ \\
    $9.$ $\neg A_2,$ $\neg i (5-7)$ \\
    $10.$ $\neg A_1 \wedge \neg A_2,$ $\wedge i (8, 9)$
}} \\

\noindent (b) We will refer to the rule we derived in (a) as follows:

$$\frac{\neg (A_1 \vee A_2)}{\neg A_1 \wedge \neg A_2} \mathcal{R}_1$$

Before proving the given sequent we will first attempt to prove the following (it will make our final proof less convoluted):

$$\neg (A_1 \to A_2) \vdash A_1 \wedge \neg A_2$$

\framebox{\parbox{\dimexpr\linewidth-0\fboxsep-0\fboxrule}{\itshape%
    $1.$ $\neg (A_1 \to A_2),$ $premise$ \\
    \framebox{\parbox{\dimexpr\linewidth-2\fboxsep-2\fboxrule}{\itshape%
        $2.$ $\neg A_1,$ $assumption$ \\
        \framebox{\parbox{\dimexpr\linewidth-2\fboxsep-2\fboxrule}{\itshape%
            $3.$ $A_1,$ $assumption$ \\
            $4.$ $\bot,$ $\neg e (2, 3)$ \\
            $5.$ $A_2,$ $\bot e (4)$
        }}
        $6.$ $A_1 \to A_2,$ $\to i (3-5)$ \\
        $7.$ $\bot,$ $\neg e (1, 6)$
    }}
    $8.$ $A_1,$ $PBC (2-7)$ \\
    \framebox{\parbox{\dimexpr\linewidth-2\fboxsep-2\fboxrule}{\itshape%
        $9.$ $A_2,$ $assumption$ \\
        $10.$ $A_1 \to A_2,$ $\to i (8, 9)$ \\
        $11.$ $\bot,$ $\neg e (1, 10)$
    }}
    $12.$ $\neg A_2,$ $\neg i (9-11)$ \\
    $13.$ $A_1 \wedge \neg A_2,$ $\wedge i (8, 12)$
}} \\

We will refer to this derived rule as follows:

$$\frac{\neg (A_1 \to A_2)}{A_1 \wedge \neg A_2} \mathcal{R}_2$$

\framebox{\parbox{\dimexpr\linewidth-0\fboxsep-0\fboxrule}{\itshape%
    \framebox{\parbox{\dimexpr\linewidth-2\fboxsep-2\fboxrule}{\itshape%
        $1.$ $\neg ((p \to q) \vee (r \to p)),$ $assumption$ \\
        $2.$ $\neg (p \to q) \wedge \neg (r \to p),$ $\mathcal{R}_1 (1)$ \\
        $3.$ $\neg (p \to q),$ $\wedge e_1 (2)$ \\
        $4.$ $\neg (r \to p),$ $\wedge e_2 (2)$ \\
        $5.$ $p \wedge \neg q,$ $\mathcal{R}_2 (3)$ \\
        $6.$ $r \wedge \neg p,$ $\mathcal{R}_2 (4)$ \\
        $7.$ $p,$ $\wedge e_1 (5)$ \\
        $8.$ $\neg p,$ $\wedge e_2 (6)$ \\
        $9.$ $\bot,$ $\neg e (7, 8)$
    }}
    $10.$ $(p \to q) \vee (r \to p),$ $PBC (1-9)$
}} \\

\noindent (c) We can use the rules derived from (a) and (b), $\mathcal{R}_1$ and $\mathcal{R}_2$, to prove the given sequent:

\framebox{\parbox{\dimexpr\linewidth-0\fboxsep-0\fboxrule}{\itshape%
    \framebox{\parbox{\dimexpr\linewidth-2\fboxsep-2\fboxrule}{\itshape%
        $1.$ $(A \to B) \to A,$ $assumption$ \\
        \framebox{\parbox{\dimexpr\linewidth-2\fboxsep-2\fboxrule}{\itshape%
            $2.$ $\neg (A \vee \neg (A \to B)),$ $assumption$ \\
            $3.$ $\neg A,$ $\mathcal{R}_1 (2)$ \\
            $4.$ $\neg \neg (A \to B),$ $\mathcal{R}_1 (2)$ \\
            $5.$ $A \to B,$ $\neg \neg e (4)$ \\
            $6.$ $A,$ $\to e (5, 1)$ \\
            $7.$ $\bot,$ $\neg e (3, 6)$
        }}
        $8.$ $A \vee \neg (A \to B),$ $PBC (2-7)$ \\
        \framebox{\parbox{\dimexpr\linewidth-2\fboxsep-2\fboxrule}{\itshape%
            $9.$ $A,$ $assumption$ \\
            $10.$ $A,$ $COPY (9)$
        }}
        \framebox{\parbox{\dimexpr\linewidth-2\fboxsep-2\fboxrule}{\itshape%
            $11.$ $\neg (A \to B),$ $assumption$ \\
            $12.$ $A \wedge \neg B,$ $\mathcal{R}_2 (11)$ \\
            $13.$ $A,$ $\wedge e_1 (12)$
        }}
        $14.$ $A,$ $\vee e (8, 9-10, 11-13)$
    }}
    $15.$ $((A \to B) \to A) \to A,$ $\to i (1-14)$
}} \\

\section*{Question 3}

\noindent (a) $\forall x. (T(x) \to \exists y. (G(y) \wedge P(y, x)))$

\noindent (b) $\forall x. (G(x) \to \exists y. (T(y) \wedge P(x, y)))$

\noindent (c) $\neg \exists x. (G(x) \wedge \exists y. (T(y) \wedge P(x, y)) \wedge \exists z. (T(z) \wedge P(x, z)) \wedge \neg (y = z))$

\noindent (d) $\exists x. (M(x, mu, rm) \wedge \neg W(mu, rm, x))$

\noindent (e) $\neg \exists x. (\neg (x = ik) \wedge P(x, mu) \wedge \forall y. ((TG(mu, y) \wedge PG(ik, y)) \to PG(x, y)))$

\noindent (f) $\forall x. ((\neg (x = ik) \wedge P(x, mu)) \to \exists y. (TG(mu, y) \wedge PG(ik, y) \wedge \neg PG(x, y)))$

\section*{Question 4}

\noindent (a)

\framebox{\parbox{\dimexpr\linewidth-0\fboxsep-0\fboxrule}{\itshape%
    $1.$ $\exists x. (P(x) \wedge Q(x)),$ $premise$ \\
    \framebox{\parbox{\dimexpr\linewidth-2\fboxsep-2\fboxrule}{\itshape%
        \, $x_0$ \\
        $2.$ $(P(x_0) \wedge Q(x_0)) = (P(x) \wedge Q(x))[x_0 / x],$ $assumption$ \\
        $3.$ $P(x')[x_0 / x'] = P(x_0),$ $\wedge e_1 (2)$ \\
        $4.$ $Q(x')[x_0 / x'] = Q(x_0),$ $\wedge e_2 (2)$ \\
        $5.$ $\exists x'. P(x'),$ $\exists i (3)$ \\
        $6.$ $\exists x'. Q(x'),$ $\exists i (4)$ \\
        $7.$ $\exists x'. P(x') \wedge \exists x'. Q(x'),$ $\wedge i (5, 6)$
    }}
    $8.$ $\exists x'. P(x') \wedge \exists x'. Q(x'),$ $\exists e (1, 2-7)$
}} \\

\noindent (b) First we will prove the following sequent:

$$\neg \exists x. A \vdash \neg A$$

\framebox{\parbox{\dimexpr\linewidth-0\fboxsep-0\fboxrule}{\itshape%
    $1.$ $\neg \exists x. A,$ $premise$ \\
    \framebox{\parbox{\dimexpr\linewidth-2\fboxsep-2\fboxrule}{\itshape%
        $2.$ $A[t / x] = A,$ $assumption$ \\
        $3.$ $\exists x. A,$ $\exists i (2)$ \\
        $4.$ $\bot,$ $\neg e (3, 1)$
    }}
    $5.$ $\neg A,$ $\neg i (2-4)$
}} \\

We will refer to this derived rule as follows:

$$\frac{\neg \exists x. A}{\neg A} \mathcal{R}_3$$

We will also make use of one of the derived rules from question 2, $\mathcal{R}_2$.

\framebox{\parbox{\dimexpr\linewidth-0\fboxsep-0\fboxrule}{\itshape%
    $1.$ $\forall x. P(x) \to S,$ $premise$ \\
    \framebox{\parbox{\dimexpr\linewidth-2\fboxsep-2\fboxrule}{\itshape%
        $2.$ $\neg \exists x. (P(x) \to S),$ $assumption$ \\
        $3.$ $\neg (P(x) \to S),$ $\mathcal{R}_3 (2)$ \\
        $4.$ $P(x) \wedge \neg S,$ $\mathcal{R}_2 (3)$ \\
        $5.$ $P(x),$ $\wedge e_1 (4)$ \\
        $6.$ $\neg S,$ $\wedge e_2 (4)$ \\
        \framebox{\parbox{\dimexpr\linewidth-2\fboxsep-2\fboxrule}{\itshape%
            \, $x_0$ \\
            $7.$ $P(x)[x_0 / x] = P(x_0),$ $COPY (5)$
        }}
        $8.$ $\forall x. P(x),$ $\forall i (7)$ \\
        $9.$ $S,$ $\to e (1, 8)$ \\
        $10.$ $\bot,$ $\neg e (9, 6)$
    }}
    $11.$ $\exists x. (P(x) \to S),$ $PBC (2-10)$
}} \\

\noindent (c)

\framebox{\parbox{\dimexpr\linewidth-0\fboxsep-0\fboxrule}{\itshape%
    $1.$ $\neg \forall x. \neg P(x),$ $premise$ \\
    \framebox{\parbox{\dimexpr\linewidth-2\fboxsep-2\fboxrule}{\itshape%
        $2.$ $\neg \exists x. P(x),$ $assumption$ \\
        \framebox{\parbox{\dimexpr\linewidth-2\fboxsep-2\fboxrule}{\itshape%
            \, $x_0$ \\
            \framebox{\parbox{\dimexpr\linewidth-2\fboxsep-2\fboxrule}{\itshape%
                $3.$ $P(x)[x_0 / x] = P(x_0),$ $assumption$ \\
                $4.$ $\exists x. P(x),$ $\exists i (3)$ \\
                $5.$ $\bot,$ $\neg e (4, 2)$
            }}
            $6.$ $(\neg P(x))[x_0 / x] = \neg P(x_0),$ $\neg i (3-5)$
        }}
        $7.$ $\forall x. \neg P(x),$ $\forall i (3-6)$ \\
        $8.$ $\bot,$ $\neg e (7, 1)$
    }}
    $9.$ $\exists x. P(x),$ $PBC (2-8)$
}} \\

\section*{Question 5}

8 hours.

\end{document}
 