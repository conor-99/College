\documentclass[12pt]{article}
\usepackage{amsmath}
\usepackage{graphicx}
\usepackage{hyperref}
\usepackage[latin1]{inputenc}
\usepackage{listings}
\renewcommand{\labelitemi}{$\textendash$}

\title{ST3009: Week 2 Assignment}
\author{Conor McCauley - 17323203}
\date{February 8, 2020}

\begin{document}

\maketitle

\section*{Question 1}

\noindent (a) Each of the 3 rolls has 6 possible outcomes.

$$ 6 ^ 3 = 216 $$

\noindent (b) For outcomes containing exactly one 2 there are 5 options each for the other rolls and ${3 \choose 2}$ ways of ordering the rolls.

$$ 5 \cdot 5 \cdot {3 \choose 2} = 75 $$

\indent For outcomes containing exactly two 2's there are 5 options for the other roll and ${3 \choose 1}$ ways of ordering the rolls. 

$$ 5 \cdot {3 \choose 1} = 15 $$

\indent There is only one outcome where all of the rolls result in a 2.

$$ \frac{75 + 15 + 1}{216} = 0.42129 $$

\noindent (c) The following code confirms the result from the previous question:

\begin{lstlisting}[language=Matlab]
    times = 1000000;
    twos = 0;
    
    for i = 1:times
        r1 = randi([1 6], 1, 1);
        r2 = randi([1 6], 1, 1);
        r3 = randi([1 6], 1, 1);
        if r1 == 2 || r2 == 2 || r3 == 2
            twos = twos + 1;
        end
    end
    
    disp(twos / times);
\end{lstlisting}

\noindent (d) The only outcomes that could sum to 17 are some ordering of 6, 6 and 5. There are ${3 \choose 2}$ orderings of these rolls.

$$ \left(\frac{1}{6}\right)^3 \cdot {3 \choose 2} = 0.013\overline{8} $$

\noindent (e) Given that the first roll was a 1 the remaining two rolls must sum to 11. This would require some ordering of 6 and 5. There are ${2 \choose 1}$ orderings of these rolls.

$$ \left(\frac{1}{6}\right)^2 \cdot {2 \choose 1} = 0.0\overline{5} $$

\section*{Question 2}

\noindent (a) The probability that the second throw is a 5 is $\frac{1}{6}$ if a six-sided die is rolled and $\frac{1}{20}$ if a 20-sided die is rolled - there is a $\frac{1}{6}$ chance that we roll the six-sided die and a $\frac{5}{6}$ chance that we roll the 20-sided die.

$$ \frac{1}{6} \cdot \frac{1}{6} + \frac{5}{6} \cdot \frac{1}{20} = 0.069\overline{4} $$

\noindent (b) The second throw can only result in a 15 if the first throw is not a 1. There is a $\frac{5}{6}$ chance that the 20-sided die is rolled and a $\frac{1}{20}$ chance that the second throw is a 15.

$$ \frac{5}{6} \cdot \frac{1}{20} = 0.041\overline{6} $$

\section*{Question 3}

\noindent I will refer to the event that a person is guilty as $G$ and the event that a person has a certain characteristic as $C$.

\begin{itemize}

    \item Likelihood - the probability that the criminal has a certain characteristic given that they are guilty:
    
    $$ P(C \mid G) = 1.0 $$
    
    \item Prior - the probability of guilt prior to receiving the new evidence:
    
    $$ P(G) = 0.6 $$
    
    \item Evidence - calculated using marginalisation:

    $$ P(C) = P(C \mid G) \cdot P(G) + P(C \mid \overline{G}) \cdot P(\overline{G}) $$
    $$ = 1.0 \cdot 0.6 + 0.2 \cdot (1 - 0.6) = 0.68 $$
    
\end{itemize}

\indent Therefore, the probability that the suspect is guilty given that they have a certain characteristic is

$$ P(G \mid C) = \frac{P(C \mid G) \cdot P(G)}{P(C)} = \frac{1.0 \cdot 0.6}{0.68} = 0.88235 $$

\section*{Question 4}

\noindent I will refer to the event that a person is in a particular location as $L$ and the event that a person is observed to be in a particular location as $O$.

\indent For each cell in the grid the probability that the phone is in the location given the two observed bars is

$$ P(L \mid O) = \frac{P(O \mid L) \cdot P(L)}{P(O \mid L) \cdot P(L) + P(O \mid \overline{L}) \cdot P(\overline{L}))} $$

\indent The following code calculates the updated probability for each cell in the grid:

\begin{lstlisting}[language=Matlab]
    pLoc = [
        0.05 0.10 0.05 0.05;
        0.05 0.10 0.05 0.05;
        0.05 0.05 0.10 0.05;
        0.05 0.05 0.10 0.05
    ];
    
    pObsLoc = [
        0.75 0.95 0.75 0.05;
        0.05 0.75 0.95 0.75;
        0.01 0.05 0.75 0.95;
        0.01 0.01 0.05 0.75
    ];
    
    % init an array of zeros
    pLocObs = zeros(4, 4);
    
    % for each grid cell calculate the value of P(L|O)
    for i = 1:numel(pLocObs)
        num = pObsLoc(i) * pLoc(i);
        den_a = (pObsLoc(i) * pLoc(i));
        den_b = ((1 - pObsLoc(i)) * (1 - pLoc(i)));
        pLocObs(i) = num / (den_a + den_b);
    end
    
    disp(pLocObs);
\end{lstlisting}

\indent It produced the following results:

$$
\begin{bmatrix}
    0.1364 &0.6786 &0.1364 &0.0028 \\
    0.0028 &0.2500 &0.5000 &0.1364 \\
    0.0005 &0.0028 &0.2500 &0.5000 \\
    0.0005 &0.0005 &0.0058 &0.1364
\end{bmatrix}
$$

\indent The code stores the known values for $P(L)$ and $P(O \mid L)$ in 4x4 arrays and initialises another array containing only zeros. Then, for each of the sixteen values, it calculates the value of $P(L \mid O)$ using the formula given at the start of my answer for this question.

\end{document}
